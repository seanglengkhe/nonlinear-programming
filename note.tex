\documentclass{article}


\usepackage{mysty}
\usepackage{amsmath, amsfonts, amssymb}
\usepackage{enumerate}
\usepackage{color}











\begin{document}





Important theorems on both linear and nonlinear programmings



\begin{enumerate}
  \item Farkas Lemma is the lemma of the alternative. 
    This lemma is focusing on the fact that there is a hyperplane that seperate a ponit from a convex cone.
    This lemma also proves a solution to a linear programming.
    For more interpretation, please refer to nonlinear programming by Mangasarian.
  \item 
\end{enumerate}



\newpage

\begin{theorem} [Farkas' Lemma] 
  Let $c \in \R^{n}$ and $A \in \R^{m \times n}$. 
  Then exactly one of the following systems has a solution:
  \begin{enumerate}[a.]
    \item $Ax \le 0, c^{\top} x >0.$
    \item $A^{\top} y = c, y \ge 0.$
  \end{enumerate}
\end{theorem}















\end{document}
